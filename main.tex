\documentclass[12pt]{article}
\usepackage[default,regular,black]{sourceserifpro}
\usepackage[T1]{fontenc}
%\usepackage{newpxtext}
%\usepackage{mathpazo}
%\usepackage{fontspec}
% \setromanfont{Times New Roman}
% \setsansfont{Arial}
 %\setmonofont[Color={0019D4}]{Courier New}

\usepackage[authoryear,round]{natbib} % Option: Use NatBib bibliography styles

\usepackage{amsmath,geometry,ulem,graphicx,caption,color,setspace,sectsty,comment,footmisc,caption,subfigure,array,longtable,rotating,booktabs,threeparttable,graphics, dirtytalk, pdflscape}

\usepackage{hyperref}
\hypersetup{
    colorlinks=true,
    citecolor=blue,
    linkcolor=red,
    filecolor=magenta,      
    urlcolor=blue,
}

\urlstyle{same}


\title{{ 
 Information Sensitivity Dynamics of Privately-Produced Safe Assets: Evidence from Subprime Securitization}\thanks{We thank Scott Frame, Kris Gerardi, Yi Li, Marco Macchiavelli, Ralf Meisenzahl,  Zhaogang Song, Larry Wall, seminar participants at the FRB of Atlanta, Tulane University, Louisiana State University, University of North Texas, and participants at the fourth annual Federal Reserve Short-Term Funding Markets Conference and 2021 FMA Conference.}
 }
\author{{ Rajesh P. Narayanan}\thanks{Louisiana State University, E.\ J.\ Ourso College of Business (\href{mailto:rnarayan@lsu.edu}{rnarayan@lsu.edu}) } \\ \\
\and
{\ Meredith E.\ Rhodes }\thanks{University of North Texas, G.\ Brint Ryan College of Business (\href{mailto:meredith.rhodes@unt.edu}{meredith.rhodes@unt.edu}) }}
\date{\today}

\begin{document}
\maketitle

\begin{abstract}
 
Subprime securitizations were designed to produce safe AAA bonds by insulating them from the risks associated with the underlying mortgages. Yet, the bonds became risky during the financial crisis of 2007-2009.  We provide evidence that following the arrival of negative public news about subprime mortgage values, investors produced private information to discriminate across deals and AAA bonds became sensitive to their underlying pool collateral.  The opacity of subprime deals amplified these effects.  Prior to the shock, AAA bonds were largely insensitive to their pool collateral.  These findings are consistent with information-based models of financial crises where negative shocks alter the information sensitivity dynamics of safe securities.

%Subprime securitizations were designed to produce safe AAA bonds by insulating them from the risks associated with the underlying mortgages. Yet, they became risky during the financial crisis of 2007-2009. We provide evidence that the decline in the ABX indices which signaled emerging risks in the subprime market altered their information sensitivity. The risk reassessment that followed caused AAA bonds to become sensitive to their own mortgage collateral, more so when it was difficult to assess loss exposure. Prior to the decline, they were informationally insensitive.  These findings are consistent with information-based models of financial crises where negative shocks incentivize private information production altering the information sensitivity of safe securities.

%Subprime securitizations were designed to produce safe AAA bonds by insulating them from the risks associated with the underlying mortgages. Yet, they became risky during the financial crisis of 2007-2009.  We provide evidence that this risk reassessment was precipitated by the decline in the ABX indices which altered the information sensitivity of ``safe" AAA bonds --- they became sensitive to their own collateral exposures, more so when it was difficult to assess their losses. Prior to the decline, they were informationally insensitive.  These findings are consistent with information-based models of financial crises where negative shocks incentivize private information production altering the information sensitivity of safe securities.

% Subprime securitizations employed dynamic credit enhancement features to insulate senior AAA tranches from risks associated with the underlying mortgages. Yet, they became risky during the financial crisis of 2007-2009. We provide evidence that when the ABX indices declined, revealing distress in the subprime market, AAA RMBS became informationally-sensitive because investors faced incentives to produce private information on their collateral exposures. Prior to the decline, they were largely insensitive.  Our findings are consistent with information based models of financial crises where the arrival of negative information incentivizes private information production about collateral values and safe debt becomes risky, marking the onset of a financial crisis.
 
 
 
 %This paper empirically shows how the information sensitivity of AAA subprime RMBS tranches evolved during the Financial Crisis of 2007--2009. When the ABX indices declined, revealing distress in the subprime market, the AAA tranches became risky and their information sensitivity changed along two dimensions. First, they became less sensitive to general information about the subprime market and more sensitive to their own risk exposures. Second, their individual risk exposures amplified their sensitivity to the ABX indices. Prior to the crisis, they were safe and largely informationally insensitive. Both these direct and amplification effects are consistent with information-based models of financial crises where negative shocks incentivize private information production about underlying collateral values.

\vspace{0in} 
\noindent \\ \textbf{Keywords:} Financial Crisis, Information sensitivity, Collateral, Opacity, Safe Assets, Subprime, Securitization \\

\end{abstract}
\thispagestyle{empty}

\clearpage
\pagenumbering{arabic} 
\doublespacing

\section{Introduction}

%The informational view of financial crises advanced by \cite{DGH2020} envisions financial crises as shifts in the information sensitivity of safe assets. According to this view, privately-produced debt securities (bank deposits, asset-backed commercial paper, repurchase agreements) are "informationally insensitive" when the cost of producing information about the underlying collateral outweighs the benefits \citep{DGH2013}. When this is the case, market participants have no incentives to produce private information, and these debt securities can trade without the fear of adverse selection.  They are safe in that their value can be taken as a given with "no questions asked," making them ``money-like'' \citep{Holmstrom2015, DGH2018}. However, when negative public news about collateral values arrives, it triggers adverse selection concerns incentivizing market participants to produce private information to discriminate across ``good'' and ``bad'' collateral.  Such private information production causes debt that was previously informationally insensitive to become informationally sensitive. The informational view characterizes such a shift in information sensitivity of safe securities as a financial crisis.


\begin{comment}


\newpage
\clearpage
\singlespacing
\bibliographystyle{aer}
\bibliography{ref2}








\end{comment}

\end{document}